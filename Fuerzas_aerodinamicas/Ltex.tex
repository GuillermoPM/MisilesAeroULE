\documentclass[a4paper]{article}
\usepackage{fancyhdr}
\rfoot[]{\thepage}
\usepackage{setspace}
\usepackage{float}
\usepackage{pythonhighlight}
\usepackage{graphicx}
\usepackage[figurename=Imagen]{caption}
\usepackage{tabto}

\title{Fuerzas aerodinámicas en el misil AIM 120 - C}
\author{Guillermo Peña Martínez \\
        Alejandro Paz Rodríguez \\
        Raúl Ordás Collado}
\date{Diciembre 2022}
\begin{document}
\maketitle
\begin{abstract}
El objetivo de este documento es realizar el análisis de resistencia aerodinámica de un misil AIM 120 C "Slammer" siguiendo los contenidos teóricos dados en la asignatura de Vehículos Lanzadores y Misiles. En los siguientes apartados se dará solución a los ejercicios planteados, indicándose el código de Python empleado para su obtención.
\end{abstract}
\section*{Glosario}
D\qquad\qquad Diámetro del misil
\singlespacing
\noindent
l\qquad\qquad Longitud de la ojiva
\section{Resistencia aerodinámica}
Las componentes de la resistencia se han dividido en cuatro subgrupos: resistencia frontal, de base, de fricción e inducida por la sustentación.
\subsection{Resistencia frontal}
Para el cálculo de esta resistencia se ha distinguido entre el drag frontal producido en la ojiva como consecuencia de la onda de choque cónica, y el generado en las alas y controles.
\subsection{Resistencia de base}
El misil a analizar presenta un tronco de cono o \textit{boattail} en la parte final del fuselaje. Considerando esta geometría, se obtiene una resistencia de base ligeramente reducida.
\subsection{Resistencia de fricción}
En la elaboración del código que ha permitido obtener esta resistencia, se han realizado una serie de consideraciones generales para poder aplicar los diferentes modelos matemáticos.
\singlespacing
\noindent
En primer lugar, la pared externa del misil se entiende aislada térmicamente, tomándose como método de cálculo el de la entalpía media, tal y como se indica en [1]. La transición de la capa límite laminar a turbulenta se ha realizado considerando el punto de paso como aquel en el que se tiene un Reynolds de $10^7$. Una vez obtenida la distancia, se ha calculado el \textit{momentum thickness} ($\theta$) en el caso laminar, para igualarlo al turbulento y tomar la distancia resultante de la ecuación que lo determina [1]:
\begin{equation}
    Cf = \frac{2\theta}{x}
\end{equation}
El Cf tomado ha sido el local, para lograr una mayor precisión, dentro de la aproximación que supone el método.
\singlespacing
\noindent
En lo relativo al código y resolución, se crearon dos funciones denominadas Laminar y Turbulenta, que permiten calcular el coeficiente de fricción local para cada punto del misil en función de la temperatura introducida. 
\section{Aceleración lateral}
Para el cálculo de la aceleración lateral, se obtuvieron los diferentes coeficientes adimensionales de las fuerzas y momentos aerodinámicos según la configuración clásica que emplea este misil. Considerando los movimientos de guiñada y alabeo desacoplados y la estabilidad del vehículo, se han tomado las ecuaciones de traslación correspondientes a la aceleración lateral y de guiñada [2]:
\begin{equation}
    \frac{dv}{dt} = Pw - Ru + gcos\theta sin\phi - \frac{C_YqS}{m}
\end{equation}
\begin{equation}
    \frac{dR}{dt} = \frac{qSl_{ref}}{I_z}(C_N-C_Y(\frac{d_{cg}-d_{cg-ref}}{l_{ref}})+C_N\frac{Rl_{ref}}{2V_m})+PQ\frac{I_x-I_y}{I_z}
\end{equation}
Entendiéndose que el misil se desplaza en un plano horizontal, estando el giro contenido en este plano, y la maniobra realizada con el motor apagado, se puede realizar la siguiente simplificación:
\begin{equation}
    \frac{dv}{dt} = - Ru  - \frac{C_YqS}{m}
\end{equation}
\begin{equation}
    \frac{dR}{dt} = \frac{qSl_{ref}}{I_z}C_N +C_N\frac{Rl_{ref}}{2I_zV_m}
\end{equation}
El efecto de la gravedad, presente por descomposición angular en la expresión de $\frac{dv}{dt}$ es despreciable por ser los ángulos $\theta$ y $\phi$ de un orden cercano a 0.
\singlespacing
\noindent
La aceleración en guiñada vendrá dada por el momento de giro aerodinámico debido a la deflexión del control, al ángulo de ataque y a la velocidad angular generada como consecuencia del giro. Se obtiene una ecuación con una única variable, R, que puede ser fácilmente despejada.
\section*{Bibliografía}
[1] Missile Aerodynamics
\singlespacing
\noindent
[2] Misiles II Tomo I
\end{document}